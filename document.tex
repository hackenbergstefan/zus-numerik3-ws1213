%% Basierend auf einer TeXnicCenter-Vorlage von Mark M�ller
%%%%%%%%%%%%%%%%%%%%%%%%%%%%%%%%%%%%%%%%%%%%%%%%%%%%%%%%%%%%%%%%%%%%%%%

% W�hlen Sie die Optionen aus, indem Sie % vor der Option entfernen  
% Dokumentation des KOMA-Script-Packets: scrguide

%%%%%%%%%%%%%%%%%%%%%%%%%%%%%%%%%%%%%%%%%%%%%%%%%%%%%%%%%%%%%%%%%%%%%%%
%% Optionen zum Layout des Artikels                                  %%
%%%%%%%%%%%%%%%%%%%%%%%%%%%%%%%%%%%%%%%%%%%%%%%%%%%%%%%%%%%%%%%%%%%%%%%
\documentclass[%
%a5paper,							% alle weiteren Papierformat einstellbar
%landscape,						% Querformat
10pt,								% Schriftgr��e (12pt, 11pt (Standard))
%BCOR1cm,							% Bindekorrektur, bspw. 1 cm
%DIVcalc,							% f�hrt die Satzspiegelberechnung neu aus
%											  s. scrguide 2.4
%twoside,							% Doppelseiten
%twocolumn,						% zweispaltiger Satz
halfparskip*,				% Absatzformatierung s. scrguide 3.1
%headsepline,					% Trennline zum Seitenkopf	
%footsepline,					% Trennline zum Seitenfu�
titlepage,						% Titelei auf eigener Seite
%normalheadings,			% �berschriften etwas kleiner (smallheadings)
%idxtotoc,						% Index im Inhaltsverzeichnis
%liststotoc,					% Abb.- und Tab.verzeichnis im Inhalt
%bibtotoc,						% Literaturverzeichnis im Inhalt
%abstracton,					% �berschrift �ber der Zusammenfassung an	
%leqno,   						% Nummerierung von Gleichungen links
%fleqn,								% Ausgabe von Gleichungen linksb�ndig
%draft								% �berlangen Zeilen in Ausgabe gekennzeichnet
DIV = 18
]
{scrartcl}

%\pagestyle{empty}		% keine Kopf und Fu�zeile (k. Seitenzahl)
%\pagestyle{headings}	% lebender Kolumnentitel  


%% Deutsche Anpassungen %%%%%%%%%%%%%%%%%%%%%%%%%%%%%%%%%%%%%
\usepackage[ngerman]{babel}
\usepackage[T1]{fontenc}
\usepackage[ansinew]{inputenc}

\usepackage{amsmath,amssymb,amsthm}

\usepackage{lmodern} %Type1-Schriftart f�r nicht-englische Texte


%% Packages f�r Grafiken & Abbildungen %%%%%%%%%%%%%%%%%%%%%%
\usepackage{graphicx} %%Zum Laden von Grafiken
%\usepackage{subfig} %%Teilabbildungen in einer Abbildung
\usepackage{hyperref}

% diese paket macht bei mir probleme, ich benutze pdfLaTeX
% hmm, wir brauchens hier ja nicht und meines Wissens ist tikz eh besser
%\usepackage{pst-all} %%PSTricks - nicht verwendbar mit pdfLaTeX


%% Boiboites %%%%%%%%%%%%%%%%%%%%%%%%%%%%%%%%%%%%%%%%%%%%%%%%%%%%%%%%%%
\usepackage{boiboites}


\newboxedtheorem[boxcolor=green, background=white, titlebackground=white,
titleboxcolor = black]{definition}{Definition}
\newboxedtheorem[boxcolor=orange, background=white, titlebackground=white,
titleboxcolor = black]{prop}{Proposition}
\newboxedtheorem[boxcolor=orange, background=white, titlebackground=white,
titleboxcolor = black]{kor}{Korollar}
\newboxedtheorem[boxcolor=blue, background=white, titlebackground=white,
titleboxcolor = black]{satz}{Satz}
\newboxedtheorem[boxcolor=red, background=white, titlebackground=white,
titleboxcolor = black]{bsp}{Beispiel}
\newboxedtheorem[boxcolor=orange, background=white, titlebackground=white,
titleboxcolor = black]{lemma}{Lemma}
\newboxedtheorem[boxcolor=purple, background=white, titlebackground=white,
titleboxcolor = black]{uebung}{�bung}
\newboxedtheorem[boxcolor=gray, background=white, titlebackground=white,
 titleboxcolor = black]{bem}{Bemerkung}

% BUG: � funktioniert nicht mit VerbatimOut
% L�sung:
\newcommand{\s}{�}
\renewcommand{\ae}{�}
\renewcommand{\oe}{�}
\newcommand{\ue}{�}
\newcommand{\Ue}{�}
\newcommand{\Ae}{�}
\newcommand{\Oe}{�}

%% Math Anpassungen %%%%%%%%%%%%%%%%%%%%%%%%%%%%%%%%%%%%%%%%%%%%%%%%%%%
\newcommand{\R}{\ensuremath \mathbb{R}}
\newcommand{\Rn}{\ensuremath \mathbb{R}^n}
\newcommand{\Rd}{\ensuremath \mathbb{R}^d}
\newcommand{\N}{\ensuremath \mathbb{N}}
\newcommand{\Q}{\ensuremath \mathbb{Q}}
\newcommand{\Z}{\ensuremath \mathbb{Z}}
\newcommand{\C}{\ensuremath \mathbb{C}}
\newcommand{\bO}{\ensuremath \overline{\Omega}}
\newcommand{\RO}{\ensuremath \partial\Omega}
\renewcommand{\L}{\ensuremath \mathcal{L}}
\newcommand{\Ccap}{\ensuremath C^2(\Omega)\cap C(\bO)}
\newcommand{\RWPh}{(RWP)$_h$~}
\newcommand{\RWP}{(RWP)~}
\newcommand{\RWPeins}{(RWP 1)~}
\newcommand{\RWPheins}{(RWP 1)$_h$~}
\newcommand{\LSGeins}{(LSG 1)~}
\newcommand{\LSG}{(LSG)~}
\DeclareMathOperator{\D}{D}


%% Andere Abk�rzungen %%%%%%%%%%%%%%%%%%%%%%%%%%%%%%%%%%%%%%%%%%%%%%%%%%
\newcommand{\zshgd}{zusammenh�ngend}



%% Bibliographiestil %%%%%%%%%%%%%%%%%%%%%%%%%%%%%%%%%%%%%%%%%%%%%%%%%%
%\usepackage{natbib}

\begin{document}

\pagestyle{empty} %%Keine Kopf-/Fusszeilen auf den ersten Seiten.


%%%%%%%%%%%%%%%%%%%%%%%%%%%%%%%%%%%%%%%%%%%%%%%%%%%%%%%%%%%%%%%%%%%%%%%
%% Ihr Artikel                                                       %%
%%%%%%%%%%%%%%%%%%%%%%%%%%%%%%%%%%%%%%%%%%%%%%%%%%%%%%%%%%%%%%%%%%%%%%%

%% eigene Titelseitengestaltung %%%%%%%%%%%%%%%%%%%%%%%%%%%%%%%%%%%%%%%    
%\begin{titlepage}
%Einsetzen der TXC Vorlage "Deckblatt" m�glich
%\end{titlepage}

%% Angaben zur Standardformatierung des Titels %%%%%%%%%%%%%%%%%%%%%%%%
%\titlehead{Titelkopf }
%\subject{Typisierung}
\title{Zusammenfassung \\ Numerik III}
% \author{Ihr Name}
%\and{Der Name des Co-Autoren}
%\thanks{Fu�note}			% entspr. \footnote im Flie�text
%\date{}							% falls anderes, als das aktuelle gew�nscht
%\publishers{Herausgeber}

%% Widmungsseite %%%%%%%%%%%%%%%%%%%%%%%%%%%%%%%%%%%%%%%%%%%%%%%%%%%%%%
%\dedication{Widmung}

\maketitle 						% Titelei wird erzeugt

%% Zusammenfassung nach Titel, vor Inhaltsverzeichnis %%%%%%%%%%%%%%%%%
%\begin{abstract}
% F�r eine kurze Zusammenfassung des folgenden Artikels.
% F�r die �berschrift s. \documentclass[abstracton].
%\end{abstract}

%% Erzeugung von Verzeichnissen %%%%%%%%%%%%%%%%%%%%%%%%%%%%%%%%%%%%%%%
\tableofcontents			% Inhaltsverzeichnis
%\listoftables				% Tabellenverzeichnis
%\listoffigures				% Abbildungsverzeichnis


%% Der Text %%%%%%%%%%%%%%%%%%%%%%%%%%%%%%%%%%%%%%%%%%%%%%%%%%%%%%%%%%%


\section{Einf�hrung}
\subsection{Klassifikation von partiellen DGLs}

\begin{definition}[Partielle DGL]{}
Sei $\Omega \subset \Rn$ offen. Eine \emph{partielle DGL} $k$-ter Ordnung hat
die Form
	\begin{equation}
		\tag{PDE ($\ast$)}
		F(x,u,\D u, \D^2 u, \dots, \D^k u) = 0\,,
	\end{equation}
	wobei
	$$F:\Omega\times \R \times \Rn \times \dots \times \R^{n^k} \to \R$$
	gegeben und $u:\Omega \to \R$ gesucht.
\end{definition}


\begin{definition}[Klassifikation PDEs]{}
	\begin{enumerate}
	  \item PDE ($\ast$) hei\s t \emph{linear}, wenn sie die Form
	  	$$\sum_{|\alpha| \leq k} a_\alpha(x) \D^\alpha u = f(x)$$
	  	hat, wobei $a_\alpha:\Omega \to \R$, $f: \Omega \to \R$ gegeben.
	  \item PDE ($\ast$) hei\s t \emph{semilinear}, wenn sie die Form
	  	$$\sum_{|\alpha| = k} a_\alpha(x) \D^\alpha u  +
	  		a_0(x,u,\D u, \dots, \D^{k-1}u) = 0$$
	  	hat, wobei $a_\alpha:\Omega \to \R$,
	  	$a_0: \Omega \times \R \times \Rn \times \dots \times \R^{n^{k-1}} \to \R$ gegeben.
	  \item PDE ($\ast$) hei\s t \emph{quasilinear}, wenn sie die Form
	  	$$\sum_{|\alpha| = k} a_\alpha(x,u,\D u,\dots, \D^{k-1}u) \D^\alpha u  +
	  		a_0(x,u,\D u, \dots, \D^{k-1}u) = 0$$
		hat, wobei $a_\alpha, a_0: \Omega \times \R \times \Rn \times \dots \times \R^{n^{k-1}} \to \R$
		gegeben.
	  \item PDE ($\ast$) hei\s t \emph{nichtlinear}, wenn sie nicht von Typ 1.--3. ist.
	\end{enumerate}
\end{definition}

% Beispiele ausgelassen

\subsection{Typeneinteilung f�r PDEs 2. Ordnung}

\begin{definition}{}
	Setze $p_i := \partial_{x_i} u$, $p_{ij} := \partial_{x_i}\partial_{x_J}u$.
	\begin{equation}
		\tag{PDE ($\ast\ast$)}
		F(x,u, p_1, \dots, p_n, p_{11}, \dots, p_{nn}) = 0
	\end{equation}
	hei\s t \emph{PDE 2. Ordnung}.
	$$M(x) := \begin{pmatrix} \partial_{p_{11}} F & \dots & \partial_{p_{1n}} F\\
		\vdots & \ddots & \vdots\\
		\partial_{p_{n1}} F & \dots & \partial_{p_{nn}} F
	\end{pmatrix}$$
\end{definition}

\begin{definition}{}
	PDE ($\ast\ast$) hei\s t
	\begin{enumerate}
	  \item \emph{elliptisch} in $x$, falls $M(x)$ positiv oder negativ definit ist.
	  \item \emph{parabolisch} in $x$, falls genau ein Eigenwert von $M(x)$ gleich 0 ist und alle anderen
	  	gleiches Vorzeichen haben.
	  \item \emph{hyperbolisch} in $x$, falls genau ein Eigenwert von $M(x)$ ein anderes Vorzeichen hat,
	  	als alle anderen.
	\end{enumerate}
\end{definition}

\begin{bsp}{}
	\begin{enumerate}
	  \item Poission-Gleichung mit $M = \begin{pmatrix}-1 &  & \\&\ddots& \\ && -1\end{pmatrix}$ ist elliptisch.
	  \item W�rmeleitungs-Gleichung mit $M = \begin{pmatrix}-1 & && \\&\ddots&& \\ && -1&\\&&& 0\end{pmatrix}$
	  	ist parabolisch.
	  \item Wellengleichung mit $M = \begin{pmatrix}-1 & && \\&\ddots&& \\ && -1&\\&&& 1\end{pmatrix}$
	  	ist hyperbolisch.
	\end{enumerate}
\end{bsp}

\section{Klassische L�sung elliptischer PDEs}


\begin{definition}[Funktionenr�ume]{}
	Sei $\Omega \subset \Rn$ offen, \zshgd, beschr�nkt.
	\begin{enumerate}
	  \item $C(\bO, \R^m)$ ist Raum aller auf $\bO$ \emph{stetigen Funktionen} nach $\R^m$.\\
	     $C(\bO) := C(\bO, \R)$.\\
	     $$\|u\|_{C(\bO,\R^m)} := \sup_{x\in \bar \Omega} \|u(x)\|\,.$$
	   \item $C^k(\bO,\R^m)$ mit $k\in\N$ ist Raum aller auf $\Omega$ \emph{$k$-mal stetig differenzierbaren
	   	Funktionen}, die zusammen mit ihren Ableitungen bis Ordnung $k$ stetig auf $\bO$ fortgesetzt werden
	   	k�nnen.
	   	$$\|u\|_{C^k(\bO,\R^m)} := \sum_{|\alpha|\leq k} \|\D^\alpha u\|_{C(\bO,\R^m)}\,.$$
	   \item $C^{0,\alpha}(\bO,\R^m) = \{u\in C(\bO,\R^m):\ \|u\|_{C^{0,\alpha}(\bO,\R^m)} < \infty\}$
	   	mit $$\|u\|_{C^{0,\alpha}(\bO,\R^m)} := \sup_{x\neq y\in \bO} \frac{\|u(x)-u(y)\|}{\|x-y\|^\alpha}\,,$$
	   	und $\alpha \in [0,1]$ ist Raum aller \emph{gleichm\ae\s ig H\oe lder stetigen Funktionen zum Exponent $\alpha$}.
	   \item $C^{k,\alpha}(\bO, \R^m) := \{u \in C^k(\bO,\R^m):\ \D^\gamma u\in C^{0,\alpha}(\bO,\R^m), |\gamma|=k\}$
	\end{enumerate}
\end{definition}

\begin{bem}{}
	$C^{k,0} = C^k$, $C^{k,1} \neq C^{k+1}$.
\end{bem}


\begin{definition}{}
	Sei $\Omega\subset \Rn$ offen, \zshgd, beschr\ae nkt. $\Omega$ geh\oe rt zur Klasse $C^{k,\alpha}$,
	$k\in \N_{0}, \alpha\in [0,1]$, wenn es endlich viele lokale Koordinatensysteme $s_1,\ldots,s_m$,
	Funktionen $\varphi_1,\ldots,\varphi_m$, sowie $r_i>0,h_i>0$, $i=1\ldots m$ existieren, sodass
	\begin{enumerate}
	  \item $\varphi_j\in C^{k,\alpha}(\bO_{r_j})$ mit $\bO_{r_j} = \left\{ y =
	  	\left[\begin{array}{c}y_1\\\vdots\\y_{n-1}\end{array}\right]:\ |y_i|\leq r_j, i=1\ldots n-1\right\}$
	  \item Zu jedem $x\in \partial \Omega$ gibt es $j\in \{1,\ldots,m\}$, sodass
	  	$x = \begin{bmatrix}y\\\varphi_j(y)\end{bmatrix}$, $y\in \Omega_{r_j}$.
	  \item $\begin{bmatrix} y \\ y_n\end{bmatrix} \in \Omega \Leftrightarrow \exists j:\ y \in \bO_{r_j}$ mit 
	  	$\varphi_j(y) < y_n < \varphi_j(y) + h_j$\\
	  	$\begin{bmatrix} y \\ y_n\end{bmatrix} \not\in \Omega \Leftrightarrow \forall j:\ y \in \bO_{r_j}$ gilt 
	  	 $\varphi_j(y)- h_j < y_n < \varphi_j(y)$
	\end{enumerate}
\end{definition}

\begin{definition}[RWP]{}
	$$\text{(RWP)}\begin{cases} \L u = f & \text{ in }\Omega \\ R u = g & \text{ auf } \partial \Omega\end{cases}$$
	mit $$\L u = -\sum_{i,j=1}^n a_{ij}(x)\partial_{x_i}\partial_{x_j} u + 
	\sum_{i=1}^n b_i(x)\partial_{x_i}u + c(x) u$$
	und $a_{ij}, b_i, c, f:\Omega \to \R$, $g:\partial \Omega \to \R$.
	Weiter $A(x) := \begin{bmatrix} a_{11}(x) & \dots & a_{1n}(x)\\ \vdots & \ddots & \vdots \\ 
		a_{n1}(x)&\dots&a_{nn}(x)\end{bmatrix}$\\
	\paragraph{Randbedingungen:}
		\begin{description}
		  \item[Dirichlet-RB] $u=g$ auf $\RO$.
		  \item[Neumann-RB] $A(x)\nabla u \cdot \nu = g$ auf $\RO$
		  \item[Robin-RB] $A(x) \nabla u \cdot \nu + \alpha(x)u = g$ auf $\RO$.
		\end{description}
	\paragraph{Voraussetzungen:}
	\begin{enumerate}
	  \item $\L$ ist gleichm\ae \s ig elliptisch, d.h. $\exists \lambda_0 > 0$, sodass
	  	$$\xi^T A(x)\xi \geq \lambda_0  \|\xi\|^2 \qquad \forall \xi \in \Rn,\ \forall x\in \Omega\,.$$
	  \item $a_{ij}, b_i, c, f \in C(\bO), g \in C(\RO)$. 
	\end{enumerate}
\end{definition}

\begin{definition}[Klassische L\oe sung]{}
	$u\in \Ccap$ hei\s t \emph{klassische L\oe sung} von (RWP) mit $Ru = u$, wenn die
	Gleichungen von (RWP) in jedem Punkt von $\Omega$ bzw. $\RO$ erf\ue llt sind.
\end{definition}


\subsection{Eindeutigkeit}
\begin{satz}[Maximums Prinzip]{2.1}
	$u\in \Ccap$ L\oe sung von (RWP) mit $f\leq 0$ in $\Omega$ und $c(x) \equiv 0$.\\
	Dann 
	$$ \sup_{x\in \bO} u(x) = \sup_{x\in \RO} u(x) = \sup_{x\in \RO} g(x)\,.$$
\end{satz}


\begin{kor}{2.2}
	Sei $c\geq 0, f\leq 0$ in $\Omega$. Dann
	$$\sup_{\bO} u \leq \sup_{\RO} \max{u,0}\,.$$
\end{kor}

\begin{kor}[Vergleichsprinzip]{2.3}
	F\ue r $u_1, u_2 \in \Ccap$ und $c\geq 0$ gelte $\L u_1 \leq \L u_2 \in \Omega$ und $u_1 \leq u_2$ auf $\RO$.\\
	Dann gilt $u_1\leq u_2$ auf $\bO$.
\end{kor}

\begin{kor}[Eindeutigkeit]{2.4}
	Sei $c\geq 0$. Dann hat (RWP) h\oe chstens eine L\oe sung $u\in \Ccap$.
\end{kor}


\subsection{Existenz}

\begin{satz}{2.5}
	$\Omega \in \Rn$ Lipschitzgebiet, $a_{ij}, b_i, c\in C(\bO)$, $c\geq 0$, $g\in C(\RO)$.\\
	Dann besitzt (RWP) genau eine L\oe sung $u\in\Ccap$.
\end{satz}


\section{Differenzenverfahren}

\subsection{Differenzenverfahren f�r die Poissongleichung in $\Omega = (a,b)$}

\begin{plainboitethm}{Differenzenverfahren}
	$$\text{\RWP* } \begin{cases} -\Delta u = f & \text{in }\Omega \\ u(a) = u_a \\ u(b) = u_b\end{cases}$$
	\paragraph{1. Diskretisierung}
	$h = \frac{b-a}{n}$, $\Omega_h = \{x_i = a+ih :\ i=1\dots n-1\}$, $\RO_h = \{x_0 = a, x_n=b\}$.
	\paragraph{2. Approximation der Ableitung}
	$$u''(x_i) \approx \frac{u(x_i+h) - 2u(x_i) + u_i(x_i - h)}{h^2} =: \Delta_hu(x_i)$$
	bezeichne $$\text{\RWPh*}\ \begin{cases}-\Delta_h u_h = f & \text{in }\Omega_h\\
		u_h = g &\text{auf }\RO_h\end{cases}$$
	\paragraph{3. Aufstellen des linearen Gleichungssystems mit OBdA $\Omega = (0,1)$}
	$$\frac{1}{h^2} \begin{bmatrix}2 & -1\\
		-1 & 2 & -1\\
		 & -1 & 2 & -1\\
		 &  &  & \ddots\\
		 &  &  & -1 & 2 & -1\\
		 &  &  &  & -1 & 2
	\end{bmatrix}
	\begin{bmatrix} u_h(x_1) \\ \\[8pt] \vdots\\[8pt]  \\ u_h(x_{n-1})\end{bmatrix}
	= 
	\begin{bmatrix} f(x_1) + \frac{g_0}{h^2} \\ f(x_2) \\ \vdots\\[-6pt]\vdots \\ f(x_{n-2})\\f(x_{n-1}) + \frac{g_1}{h^2}
	\end{bmatrix}$$
	bezeichne $\text{(LSG 1)}\ -\tilde\Delta_u \tilde u_h = \tilde f_h$.
\end{plainboitethm}


\begin{definition}[Konvergenzordnung]{}
	$U_h:=\{f: \Omega_h \to \R\}, R_h: C(\bO)\to U_h$ die Einschr\ae nkung. (RWP)$_h$ hei\s t \emph{konvergent von Ordnung $P$},
	falls $\exists c>0, h_0> 0$, sodass f�r die L\oe sung $u$ von \RWP und die L\oe sung $u_h$ von \RWPh gilt:
	$$\|u_h - R_h u\|_h \leq ch^P\qquad \forall h\leq h_0\,.$$
\end{definition}

\begin{definition}[Konsistenzordnung]{}
	\RWPh hei\s t \emph{konsitent von Ordnung $P$}, falls
	$$\| \L_h R_hu - R_h \L u\|_h \leq c \|u\|_{C^{P+2}(\bO)} h^P \qquad \forall u\in C^{p+2}(\bO)\,.$$
\end{definition}

\begin{definition}[Stabil]{}
	\RWPh hei\s t \emph{stabil}, falls $\tilde\L_h$ invertierbar und $\exists h_0 > 0:$
	$$\sup_{0<h\leq h_0}\|\tilde\L_h^{-1}\| < \infty\,.$$ 
\end{definition}


\begin{satz}{3.1}
	\RWPh konsistent und stabil. Dann konvergent.\\
	Ist es zus\ae tzlich konsistent von Ordnung $P$ und $u\in C^{p+1}(\bO)$. Dann konvergent von Ordnung $P$. 
\end{satz}

\clearpage
\section{Alle Definitionen}
\printalldefinition

\clearpage
\section{Alle S�tze}
\printallsatz

\end{document}
