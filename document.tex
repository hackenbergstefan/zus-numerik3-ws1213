%% Basierend auf einer TeXnicCenter-Vorlage von Mark M�ller
%%%%%%%%%%%%%%%%%%%%%%%%%%%%%%%%%%%%%%%%%%%%%%%%%%%%%%%%%%%%%%%%%%%%%%%

% W�hlen Sie die Optionen aus, indem Sie % vor der Option entfernen  
% Dokumentation des KOMA-Script-Packets: scrguide

%%%%%%%%%%%%%%%%%%%%%%%%%%%%%%%%%%%%%%%%%%%%%%%%%%%%%%%%%%%%%%%%%%%%%%%
%% Optionen zum Layout des Artikels                                  %%
%%%%%%%%%%%%%%%%%%%%%%%%%%%%%%%%%%%%%%%%%%%%%%%%%%%%%%%%%%%%%%%%%%%%%%%
\documentclass[%
%a5paper,							% alle weiteren Papierformat einstellbar
%landscape,						% Querformat
10pt,								% Schriftgr��e (12pt, 11pt (Standard))
%BCOR1cm,							% Bindekorrektur, bspw. 1 cm
%DIVcalc,							% f�hrt die Satzspiegelberechnung neu aus
%											  s. scrguide 2.4
%twoside,							% Doppelseiten
%twocolumn,						% zweispaltiger Satz
halfparskip*,				% Absatzformatierung s. scrguide 3.1
%headsepline,					% Trennline zum Seitenkopf	
%footsepline,					% Trennline zum Seitenfu�
titlepage,						% Titelei auf eigener Seite
%normalheadings,			% �berschriften etwas kleiner (smallheadings)
%idxtotoc,						% Index im Inhaltsverzeichnis
%liststotoc,					% Abb.- und Tab.verzeichnis im Inhalt
%bibtotoc,						% Literaturverzeichnis im Inhalt
%abstracton,					% �berschrift �ber der Zusammenfassung an	
%leqno,   						% Nummerierung von Gleichungen links
%fleqn,								% Ausgabe von Gleichungen linksb�ndig
%draft								% �berlangen Zeilen in Ausgabe gekennzeichnet
DIV = 18
]
{scrartcl}

%\pagestyle{empty}		% keine Kopf und Fu�zeile (k. Seitenzahl)
%\pagestyle{headings}	% lebender Kolumnentitel  


%% Deutsche Anpassungen %%%%%%%%%%%%%%%%%%%%%%%%%%%%%%%%%%%%%
\usepackage{amsmath}
\usepackage[ngerman]{babel}
\usepackage[T1]{fontenc}
\usepackage[ansinew]{inputenc}

\usepackage{lmodern} %Type1-Schriftart f�r nicht-englische Texte


%% Packages f�r Grafiken & Abbildungen %%%%%%%%%%%%%%%%%%%%%%
\usepackage{graphicx} %%Zum Laden von Grafiken
%\usepackage{subfig} %%Teilabbildungen in einer Abbildung
\usepackage{hyperref}

% diese paket macht bei mir probleme, ich benutze pdfLaTeX
%\usepackage{pst-all} %%PSTricks - nicht verwendbar mit pdfLaTeX


%% Boiboites %%%%%%%%%%%%%%%%%%%%%%%%%%%%%%%%%%%%%%%%%%%%%%%%%%%%%%%%%%
\usepackage{boiboites}


\newboxedtheorem[boxcolor=green, background=white, titlebackground=white,
titleboxcolor = black]{definition}{Definition}
\newboxedtheorem[boxcolor=orange, background=white, titlebackground=white, 
titleboxcolor = black]{prop}{Proposition}
\newboxedtheorem[boxcolor=orange, background=white, titlebackground=white, 
titleboxcolor = black]{kor}{Korollar}
\newboxedtheorem[boxcolor=blue, background=white, titlebackground=white, 
titleboxcolor = black]{satz}{Satz}
\newboxedtheorem[boxcolor=red, background=white, titlebackground=white, 
titleboxcolor = black]{bsp}{Beispiel}
\newboxedtheorem[boxcolor=orange, background=white, titlebackground=white, 
titleboxcolor = black]{lemma}{Lemma}
\newboxedtheorem[boxcolor=purple, background=white, titlebackground=white,
titleboxcolor = black]{uebung}{�bung}
\newboxedtheorem[boxcolor=gray, background=white, titlebackground=white,
 titleboxcolor = black]{bem}{Bemerkung}


%% Bibliographiestil %%%%%%%%%%%%%%%%%%%%%%%%%%%%%%%%%%%%%%%%%%%%%%%%%%
%\usepackage{natbib}

\begin{document}

\pagestyle{empty} %%Keine Kopf-/Fusszeilen auf den ersten Seiten.


%%%%%%%%%%%%%%%%%%%%%%%%%%%%%%%%%%%%%%%%%%%%%%%%%%%%%%%%%%%%%%%%%%%%%%%
%% Ihr Artikel                                                       %%
%%%%%%%%%%%%%%%%%%%%%%%%%%%%%%%%%%%%%%%%%%%%%%%%%%%%%%%%%%%%%%%%%%%%%%%

%% eigene Titelseitengestaltung %%%%%%%%%%%%%%%%%%%%%%%%%%%%%%%%%%%%%%%    
%\begin{titlepage}
%Einsetzen der TXC Vorlage "Deckblatt" m�glich
%\end{titlepage}

%% Angaben zur Standardformatierung des Titels %%%%%%%%%%%%%%%%%%%%%%%%
%\titlehead{Titelkopf }
%\subject{Typisierung}
\title{Zusammenfassung \\ Numerik III}
% \author{Ihr Name}
%\and{Der Name des Co-Autoren}
%\thanks{Fu�note}			% entspr. \footnote im Flie�text
%\date{}							% falls anderes, als das aktuelle gew�nscht
%\publishers{Herausgeber}

%% Widmungsseite %%%%%%%%%%%%%%%%%%%%%%%%%%%%%%%%%%%%%%%%%%%%%%%%%%%%%%
%\dedication{Widmung}

\maketitle 						% Titelei wird erzeugt

%% Zusammenfassung nach Titel, vor Inhaltsverzeichnis %%%%%%%%%%%%%%%%%
%\begin{abstract}
% F�r eine kurze Zusammenfassung des folgenden Artikels.
% F�r die �berschrift s. \documentclass[abstracton].
%\end{abstract}

%% Erzeugung von Verzeichnissen %%%%%%%%%%%%%%%%%%%%%%%%%%%%%%%%%%%%%%%
\tableofcontents			% Inhaltsverzeichnis
%\listoftables				% Tabellenverzeichnis
%\listoffigures				% Abbildungsverzeichnis


%% Der Text %%%%%%%%%%%%%%%%%%%%%%%%%%%%%%%%%%%%%%%%%%%%%%%%%%%%%%%%%%%


\section{Einf�hrung}
\subsection{Klassifikation von partiellen DGLs}

\begin{definition}[Partielle DGL]{}
	Sei \ldots
\end{definition}


\end{document}
