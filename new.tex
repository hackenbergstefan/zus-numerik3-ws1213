\begin{lemma}{4.11}
  Falls $b(x)=0$ und $c(s)\geq 0$ in $\Omega$, dann ist $B$ in $(VGL1)'$
  koerzitiv.
\end{lemma}

\begin{satz}{4.12}
  Sei $\Omega\subset\R^d$ beschr\ae nkt, und sei $\cL u=-div(A(x)Du)+c(x)U$
  gleichm\ae \s ig elliptisch, $c(x)\geq 0$ in $\Omega$,
  $a_{ij},c\in\mathcal{L}(\Omega)$, $f\in\mathcal{L}^2(\Omega)$. Dann besitzt
  $(VGL1)$ eine Eindeutige L\oe sung $u\in H^1_0(\Omega)$. Au\s erdem existiert eine
  Konstante $c>0$ so dass $||u||_{H^1(\Omega)}\leq\hat c
  ||f||_{\mathcal{L}^2(\Omega)}$.
\end{satz}

\begin{definition}[alternative def f\ue r schwach Lsg]{}
  Sei $f\in H^{-1}(\Omega)$ Eine Funktion $u\in H^1_0(\Omega)$ hei\s t schwache
  L\oe sung von $\RWPeins$, falls $B(u.\phi)=<f,\phi>_{H^{-1},H^1_0}$
  $\forall\phi\in H^1_0(\Omega)$.
\end{definition}

\begin{bem}{}
  Es existiert ein $\mu_u>0$, so dass $\forall \mu>\mu_0$ das RWP
  \[\mathcal{L}u-\mu u = f\mbox{ in } \Omega\]
  \[u=0\mbox{ auf }\partial\Omeag\]
  f\ue r alle $f\in H^{-1}(\Omega)$ eine eindeutige schwache L\oe sung
  $u\in H^1_0(\Omega)$ besitzt.
\end{bem}

\begin{lem}[Young-Ungl]{}
  f\ue r $\alpha,\beta>0$, $\epsilon>0$ gilt:
  \[\alpha\beta \leq \epsilon\alpha^2+\frac{1}{4\epsilon}\beta^2\]
\end{lem}

%4.4
\subsection{Randwertprobleme mit inhomogenen Dirichlet-RB und/oder Robin-RB}
\begin{enumerate}
  \item Inhomogene Dirichlet-RB\\
    \[(RWP2)\begin{cases}
    \cL u=f & \mbox{ in }\Omega\\
    u=f & \mbox{ auf }\partial\Omega
    \end{cases}\]

    $g:\partial\Omega \rightarrow \RR$ und es existiert stetige Fortsetzung
    $\tilde g:\Omega\rightarrow \RR$ mit $\tilde g|_{\parital\Omega}=g$

    \textbf{Superpositionsprinzip:} $u\in\C^2(\Omega)\cap C(\bar\Omega)$ l\oe
    st $(RWP2)$ $\Leftrightarrow$
    $v=u-\tilde g$ l\oe st das $RWP$
    \[\begin{cases}
    \cL v=f-\cL\tilde g & \mbox{ in }\Omega\\
    v=0 & \mbox{ auf }\partial\Omega
    \end{cases}\]

    \textbf{Schwach Formulierung}:
    \[\int_\Omega Dv\cdot D\phi+b(x)\cdot Dv\phi +c(x)v\phi dx =\int_\Omega
    f\phi dx - \intA(x)D\tilde g \cdot D\phi + b(x)\cdot D\tilde g \phi dx\]
    $\forall \phi \in H^1_0(\Omega)$ und gesucht ist: $v\in H^1_0(\Omega)$\\
    $\Rightarrow$ $u=v+\tilde g$\\
    $\Rightarrow$ $(VLG 2)$: $\int_\Omega A(x)Du\cdot D\phi+b(x)\cdot Du \phi+ 
    cu\phi dx = \int_\Omega f\phi dx$ $\forall \phi\in H^1_0(\Omega)$ 
    (Testraum)\\
    $u\in \{ w\in H^1(\Omega) : \tau(w)=g \}$ (L\oe sungsraum) wobei
    $tau:H^1(\Omega)\rightarrow \cL^2(\partial\Omega);
    \tau(w)=w|_{\partial\Omega}$ $\forall w\in H^1(\Omega)\cap C(\bar \Omega)$
  \item Inhomogen Robin-RB\\
    \[\begin{cases}
    \cL u=f & \mbox{ in }\Omega\\
      A(x)Du\cdot \nu+\mu u=g & \mbox{ auf }\partial\Omega
    \end{cases}\]
    betrachte nun als Testfunktionen die $\phi\in C^\infty(\bar\Omega)$

\end{enumerate}



% vim: set ft=tex :
